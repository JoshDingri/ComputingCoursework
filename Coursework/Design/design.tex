\chapter{Design}

\section{Overall System Design}

\subsection{Short description of the main parts of the system}

Hardware Allocation Database

\begin{itemize}
\item IT staff user interface
\item Line manager user inferface
\item Staff user interface
\item Login screen
\item Viewing Databases
\item Editing Databases
\end{itemize}

\

\textbf{IT staff user interface}

\begin{itemize}
\item IT staff have admin rights to the system. They will be able to view everyones information and are the only people who may edit information on the system.
\item Once logged in they will be presented with a user interface with buttons allowing them to open a database or search for staff.
\item All IT staff will share a username and password.
\item The search function will take the user to an interface which will allow them to search for a staff member and be shown which hardware devices they own. This search could also be performed by going to the staff database and using the search function there, but this may be used frequently and so is placed here for convience.
\item The open database button will take them to a user interface that will have a dropdown box allowing them to pick which database to view. There will also be a "Continue" button to click when the database is selected and the database will be presented to them. There will also be a "Edit Database" button to edit, add or remove entries. A search button will also be on this screen to allow the user to search for a specific field.
\item The edit database will allow the user to change information on the database.
\end{itemize}

\

\textbf{Line manager user interface}

\begin{itemize}
\item Line manager's have more rights than normal staff but unlike IT staff they cannot edit any information.
\item Line manager's share the same username if they are part of the same department (for example someone from Liverpool Financing will use the same username as someone from Orwell Financing) and a password that they may change at any time. Upon first using the system their password will be issued by IT staff. Their username will be a 5 digit number allocated to them by IT staff (the best way to do this will be via email). 
\item Line manager's can view information of members from their own department.
\item Once logged in, the user interface will show a "View Department" button and a "View My Information" button.
\item The "View Department" button will show the staff database but only show records from staff at their own department. There will be a search function to find a specific field.
\item The "View My Information" button will allow the user to view their own hardware devices and personal information. It may be used often to check the warranty period of an item and to check correct information is being displayed.
\end{itemize}

\

\textbf{Staff User Interface}
\begin{itemize}
\item Ordinary staff members will all have unique usernames and be able to create their own password.
\item After logging on they will be taken to a user interface with a button called "View My Information".
\item The "View My Information" button will allow the user to view their own hardware devices and personal information. It may be used often to check the warranty period of an item and to check correct information is being displayed.
\end{itemize}

\

\textbf{Login screen}
\begin{itemize}
\item The login screen will have a username field and a password field. There will also be two buttons, one to login and the other will be "Forgot Your Password?".
\item The "Forgot Your Password?" button will take the user to another interface which will ask for their Volac (company) email address, Forename and Surname. This information will be sent to the IT staff who can review it and email the user another password to use or tell them what their password was. It is important to note that all Volac staff have the same buisness email (for example XXX.Volac.com), this means SMTP settings can be set up easily in Python. This process will be easier than asking IT Staff directly.
\item Normal staff all have their own username and a password that they may change at any time. Upon first using the system their password will be issued by IT staff. Their username will be a 5 digit number allocated to them by IT staff (the best way to do this will be via email). 
\item Line manager's share the same username if they are part of the same department (for example someone from Liverpool Financing will use the same username as someone from Orwell Financing) and a password that they may change at any time. Upon first using the system their password will be issued by IT staff. Their username will be a 5 digit number allocated to them by IT staff (the best way to do this will be via email). 
\item All IT staff will share a username and password to have admin rights.
\item There are 100,000 combinations on a 5 digit number (10$^5$) which will be more than enough for the company, it will be best to start from the left and issue 10000 to IT staff and start with 20000 for other staff (and proceed with 20001, 20002, ... 20010...)
\end{itemize}

\

\textbf{Viewing Databases}

\begin{itemize}
\item All staff, in some way, will be able to view a database.
\item For general staff, they will only be able to view their own data. Line Manager's will be able to view all data in the staff database about the members in their department. IT staff have full access to all databases to view all information.
\item A dropdown box will be useful for IT staff to pick a database to view, buttons will surffice for other staff since they can only view a couple of databases.
\end{itemize}

\

\textbf{Editing Databases}
\begin{itemize}
\item To add entries the user will click the "Edit Database" button (found on the IT staff interface).
\item This button will then allow the user to either edit existing data or take the user to a form which allows data to be input into fields for such things as First Name, Surname, Warranty or Hardware Device.
\item Validation will be required for various fields such as telephone numbers (string of length 11) and dropdown boxes will be used for boolean values for fields such as Warranty.
\end{itemize}


\subsection{System flowcharts showing an overview of the complete system}

\begin{figure}[H]
\includegraphics[width=.9\textwidth,height=.9\textheight,keepaspectratio]{FlowchartKey.jpg}
\end{figure}

\newpage

\begin{figure}[H]
\includegraphics[width=\textwidth]{FlowchartPart1.jpg}
\end{figure}

\begin{figure}[H]
\includegraphics[width=\textwidth]{FlowchartPart2.jpg}
\end{figure}

\begin{figure}[H]
\includegraphics[width=\textwidth]{FlowchartPart3.jpg}
\end{figure}

\begin{figure}[H]
\includegraphics[width=\textwidth]{FlowchartPart4.jpg}
\end{figure}

\begin{figure}[H]
\includegraphics[width=\textwidth]{FlowchartPart5.jpg}
\end{figure}



\section{User Interface Designs}

\begin{figure}[H]
\includegraphics[width=\textwidth]{GUI_Design1.jpg}
\caption{}
\end{figure}

\begin{figure}[H]
\includegraphics[width=\textwidth]{GUI_Design2.jpg}
\caption{}
\end{figure}

\begin{figure}[H]
\includegraphics[width=\textwidth]{GUI_Design3.jpg}
\caption{}
\end{figure}

\begin{figure}[H]
\includegraphics[width=\textwidth]{GUI_Design4.jpg}
\caption{}
\end{figure}

\begin{figure}[H]
\includegraphics[width=\textwidth]{GUI_Design5.jpg}
\caption{}
\end{figure}

\begin{figure}[H]
\includegraphics[width=\textwidth]{GUI_Design6.jpg}
\caption{}
\end{figure}

\begin{figure}[H]
\includegraphics[width=\textwidth]{GUI_Design7.jpg}
\caption{}
\end{figure}

\begin{figure}[H]
\includegraphics[width=\textwidth]{GUI_Design8.jpg}
\caption{}
\end{figure}

\begin{figure}[H]
\includegraphics[width=\textwidth]{GUI_Design9.jpg}
\caption{}
\end{figure}

\begin{figure}[H]
\includegraphics[width=\textwidth]{GUI_Design10.jpg}
\caption{}
\end{figure}

\begin{figure}[H]
\includegraphics[width=\textwidth]{GUI_Design11.jpg}
\caption{}
\end{figure}

\begin{figure}[H]
\includegraphics[width=\textwidth]{GUI_Design12.jpg}
\caption{}
\end{figure}

\begin{figure}[H]
\includegraphics[width=\textwidth]{GUI_Design13.jpg}
\caption{}
\end{figure}

\begin{figure}[H]
\includegraphics[width=\textwidth]{GUI_Design14.jpg}
\caption{}
\end{figure}


\section {Hardware Specification}

The system will be stored onto a server stored in the workplace which is perfectly capable to handle the system, the specs are as followed:
\begin{itemize}
\item HP DL360
\item Windows 2003 (can run any operating system required)
\item 4TB Hard Drive
\item 16GB RAM
\item Quad Core Processor - 2.5 GHZ
\end{itemize}
The server is stored in a server room with high ventilation and fans operated 24/7. This is a huge benefit because the system can be running at all times so people can access the database. Preferably the overall model will be client-server and each user will have their own login for security. All users should connect using clients on local computers and will not directly access the server.

Users will connect to the system using their own computers at the workplace. These computers all have Windows 7 installed and run at the resolution of 1920x1080. The monitor sizes range from different locations, but the smallest would be 21" LCD monitors and the largest would be 27". These sizes will not be a problem since the application can fit on these monitors. All computers have a mouse, which is required for clicking the interface buttons, and a keyboard which is required for entering information. This system will not be developed for touch screen devices. The data for the program will be held on a hard drive inside the server that can be accessed by everyone who is connected to it. The company will not need any additional hardware to run the proposed system.

\section{Program Structure}

\subsection{Top-down design structure charts}

\begin{figure}[H]
\includegraphics[width=\textwidth]{PS.jpg}
\caption{}
\end{figure}

\begin{figure}[H]
\includegraphics[width=\textwidth]{PS1.jpg}
\caption{}
\end{figure}

\begin{figure}[H]
\includegraphics[width=\textwidth]{PS3.jpg}
\caption{}
\end{figure}

\begin{figure}[H]
\includegraphics[width=\textwidth]{PS4.jpg}
\caption{}
\end{figure}

\begin{figure}[H]
\includegraphics[width=\textwidth]{PS5.jpg}
\caption{}
\end{figure}

\begin{figure}[H]
\includegraphics[width=\textwidth]{PS6.jpg}
\caption{}
\end{figure}

\begin{figure}[H]
\includegraphics[width=\textwidth]{PS7.jpg}
\caption{}
\end{figure}
\begin{figure}[H]
\includegraphics[width=\textwidth]{PS8.jpg}
\caption{}
\end{figure}

\subsection{Algorithms in pseudo-code for each data transformation process}

\subsection{Object Diagrams}

\begin{figure}[H]
\hspace*{-1.3cm}
\vspace*{-1cm}
\includegraphics[width=1\textwidth]{RelationshipDiagram.jpg}
\caption{The relationship diagram shows how staff own many hardware devices. The staff work at one work location and that location has many different departments. Many staff are part of one department. } \label{Relationship Diagrams}
\end{figure}

\subsection{Class Definitions}

\begin{figure}[H]
\hspace*{-1.3cm}
\vspace*{-1cm}
\includegraphics[width=1\textwidth]{ClassDefinitions.jpg}
\caption{The class definitions show the attributes and methods for each class. The staff class will hold unique IDs from the other classes to make the relationship.} \label{Class Definitions}
\end{figure}

\section{Prototyping}

Prototyping is going to be very helpful as I can test out different parts of my program and see if they will function the way I would like them to or not.

\

\underline {\textbf{The Login Window}}

\

I made a prototype for the login window to get to grips with how to validate information and use colours. The validation tells the user that the username has to be 5 digits (cannot include letters). The password is the part of the prototype that was focused on most since it is important that it is censored out. The password cannot be copied and pasted anywhere (to add security).

\begin{figure}[H]
\includegraphics[width=\textwidth]{LoginWindow.jpg}
\caption{Login Window}
\end{figure}

\newpage

The login window will have a system that tells the user a correct format has been added (A tick will be placed next to the box).
\begin{figure}[H]
\includegraphics[width=\textwidth]{LoginWindow2.jpg}
\caption{Login Window}
\end{figure}

\newpage

\underline {\textbf{The Calender}}

\

Another prototype I made was a calender. This will be used when inputting dates. I chose to design this to see how difficult it would be to track when someone clicks a date on the calender and display the date that was clicked. For example below shows a label at the bottom displaying what date has been selected.

\begin{figure}[H]
\includegraphics[width=\textwidth]{Calender.jpg}
\caption{Login Window}
\end{figure}


\section{Definition of Data Requirements}

\subsection{Identification of all data input items}

\begin{itemize}
\item Username
\item Password
\item Email Address
\item First Name
\item Surname
\item Job Title
\item Date (On Bug Report)
\item Details of Bug Reports
\item Details of Information Errors
\item Department
\item Location
\item Hardware Cost
\item Hardware Warranty
\item Hardware Serial Number
\item Hardware IMEI Number
\item Phone Number
\item Hardware Make
\item Hardware Model
\item Device Name
\item Purchase Date
\item Location Name 
\item Location Address Line 1
\item Location Address Line 2
\item Location Address Line 3
\end{itemize}

\subsection{Identification of all data output items}

\underline {\textbf{Information that outputs to database}:}

\begin{itemize}
\item Staff First Name
\item Staff Surname
\item Staff Job Title
\item Department
\item Location
\item Hardware Cost
\item Hardware Warranty
\item Hardware Serial Number
\item Hardware IMEI Number
\item Phone Number
\item Hardware Make
\item Hardware Model
\item Device Name
\item Purchase Date
\item Location Name 
\item Location Address Line 1
\item Location Address Line 2
\item Location Address Line 3
\end{itemize}

\subsection{Explanation of how data output items are generated}

\begin{center}
    \begin{tabular}{|p{5cm}|p{5cm}|}
        \hline
        \textbf{Output} & \textbf{How Output is Generated}\\ \hline
	Purchase Date & When the calender date is clicked for data inputs the date will automatically enter.\\ \hline
	Staff First Name & IT Staff Input Information \\ \hline
	Staff Last Name & IT Staff Input Information \\ \hline
	Staff Job Title & IT Staff Input Information \\ \hline
	Department & IT Staff Input Information \\ \hline
	Location & IT Staff Input Information \\ \hline
	Hardware Cost & IT Staff Input Information \\ \hline
	Hardware Warranty & IT Staff Input Information \\ \hline
	Hardware Serial Number & IT Staff Input Information \\ \hline
	Hardware IMEI Number & IT Staff Input Information \\ \hline
	Phone Number & IT Staff Input Information \\ \hline
	Hardware Make & IT Staff Input Information \\ \hline
	Hardware Model & IT Staff Input Information \\ \hline
	Device Name & IT Staff Input Information \\ \hline
	Purchase Date & IT Staff Input Information \\ \hline
	Location Name & IT Staff Input Information \\ \hline
	Location Address Line 1 & IT Staff Input Information \\ \hline
	Location Address Line 2 & IT Staff Input Information \\ \hline
	Location Address Line 3 & IT Staff Input Information \\ \hline
    \end{tabular}
\end{center}

\newpage

\subsection{Data Dictionary}

\begin{center}
\begin{longtable}{|p{2cm}|p{1.14cm}|p{1.1cm}|p{1.7cm}|p{1.7cm}|p{2cm}|}
\hline
\textbf{Name} & \textbf{Data Type}& \textbf{Length} & \textbf{Validation} & \textbf{Example Data} & \textbf{Comment}      \\ \hline
StaffID                             & Integer                                 & 1-200                     & Range Validation \textgreater0 and \textless=350                                 & 132                   & Unique to the staff   \\ \hline
Staff-FirstName                      & String                                  & 1-25                                 & Presence Check                           & John                  &                       \\ \hline
Staff-LastName                       & String                                  & 1-25                                 & Presence Check                           & Smith                 &                       \\ \hline
StaffJobTitle			& String				& 1-30			& Length			& Finance Manager		&		\\ \hline
HardwareID                          & Integer                                 & 1-200                                & Range Validation \textgreater0 and  \textless=300                      & 121                   & Unique to each device \\ \hline
Hardware Cost                       & Float                                 & 1-2000                             & Range Validation \textgreater0 and \textless=2000                                    & £500                  &                       \\ \hline
Hardware Warranty                    & Boolean                                 &                                      & Presence Check                           & True                  &                       \\ \hline
Hardware WarrantyPeriod              & String                                  & 1-10                                 & Length                                   & 3 Years               &                       \\ \hline
Hardware SerialNumber                & String                                  & 1-50                                 & Length                                   & 12307321              &                       \\ \hline
Hardware PurchaseDate                & Date                                  &                                  & Format                                   & 11/03/2015              &                       \\ \hline
Hardware-IMEI                & String                                  &          1-50                        & Length                                   & EH37000781               & Only required for phones                      \\ \hline
Hardware PhoneNumber                & String                                  &1-11                                  & Length                                   & 07927551125              &   Only required for phones                      \\ \hline
DeviceID                      & Integer                                  & 1-200                                & Range Validation \textgreater0 and \textless=100                                   & 24                 & Unique to each device                       \\ \hline
DeviceName                        & String                                  & 1-25                                 & Length                                   & Phone, Laptop, Tablet                &                       \\ \hline
Hardware MakeID                      & Integer                                  & 1-200                                & Range Validation \textgreater0 and \textless=100                                   & 24                 & Unique to each make                       \\ \hline
Hardware Make                        & String                                  & 1-25                                 & Length                                   & iPhone                &                       \\ \hline
Hardware ModelID		& Integer                                  & 1-200                                & Range Validation \textgreater0 and \textless=100                                   & 24                 & Unique to each model                       \\ \hline
Hardware Model                       & String                                  & 1-25                                 & Length                                   & 5S                    &                       \\ \hline
DepartmentID &Integer				& 1-4				&Range Validation \textgreater0 and \textless=75		& 24		& Unique for each department	\\ \hline
Department Name & String				& 1-30				& Length		& Financing		&	\\ \hline
LocationID &Integer				& 1-4				&Range Validation \textgreater0 and \textless=20		& 	24	& Unique for each work location		\\ \hline
Location Name & String				& 1-30				& Length		&	Orwell	 &  \\ \hline
Location AddrLine1 & String				& 1-30				& Length		&	50 Fisher's Ln	 &  \\ \hline
Location AddrLine2 & String				& 1-30				& Length		&	Orwell	 &  \\ \hline
Location AddrLine3  & String				& 1-30				& Length		&	Royston 	 &  \\ \hline
\end{longtable}
\end{center}

\newpage
\subsection{Identification of appropriate storage media}

This database will be accessed by a large amount of computers at once and it will need to be updated everytime someone changes it. This would be impossible to do if it was stored locally on one machine since the changes will only be made on that one machine. Therefore the best way to do this would be to store the system on the company's server. The company already have a large number of servers in their ventilated server room and using one to store the system on is not a problem. Every computer in the company has access to the server so making changes can be done on any machine and viewed from any other computer. The data will be stored on a 4TB hard drive inside the server and backups can be made on backup drives the company hold which are 2-4TB in size. The company mainly uses online backups through cloud storage which can be used as well, the file size plays little impact as their download speed is over 100mb/s with an upload speed of about 70mb/s so the system can be backed up quickly.
\section{Database Design}

\subsection{Normalisation}

\newpage

\subsubsection{ER Diagrams}

\begin{figure}[H]
\includegraphics[width=\textwidth]{ERNormalisedDiagram.jpg}
\end{figure}


\subsubsection{Entity Descriptions}


\textbf{Staff}  (\underline{StaffID}, FirstName, LastName, JobTitle, \textit{DepartmentID},\\ \textit{ LocationID})


\

\textbf{Hardware}  (\underline{HardwareID}, \textit{DeviceID},  \textit{HardwareModelID},\\ HardwareCost, HardwareWarranty, HardwareWarrantyPeriod,\\ HardwareSerialNumber, 		HardwareIMEINumber, \\HardwarePhoneNumber)

\

\textbf{HardwareMake} (\underline{HardwareMakeID}, HardwareMakeName)

\

\textbf{HardwareModel} (\underline{HardwareModelID}, HardwareModelName, \textit{HardwareMakeID})

\


\textbf{DeviceType}  (\underline{DeviceID}, DeviceName)


\


\textbf{StaffHardware}  (\underline{ StaffID}, \underline{ HardwareID},PurchaseDate)


\


\textbf{Department}  (\underline{DepartmentID}, DepartmentName)


\


\textbf{Location}  (\underline{LocationID}, LocationName, LocationAddrLine1, LocationAddrLine2, LocationAddrLine3)

\

\textbf{DepartmentLocation} (\underline{LocationID}, \underline{DepartmentID})




\subsubsection{1NF to 3NF}

\begin{figure}[H]
\includegraphics[width=\textwidth]{UNF&1NF.jpg}
\end{figure}

\begin{figure}[H]
\includegraphics[width=\textwidth]{2NF.jpg}
\end{figure}

\begin{figure}[H]
\includegraphics[width=\textwidth]{3NF.jpg}
\end{figure}

\subsection{SQL Queries}

\begin{center}
\begin{tabular}{|p{6cm}|p{5cm}|}
\hline
\textbf{SQL}      & \textbf{Description} \\ \hline
"""create table Staff(\

   StaffID INTEGER,\

   FirstName TEXT,\

   LastName TEXT,\

   JobTitle TEXT,\

   DepartmentID INTEGER,\

   LocationID INTEGER,\

   PRIMARY KEY(StaffID))\

   FOREIGN KEY(DepartmentID) \

   REFERENCES Department(DepartmentID)\

   FOREIGN KEY(LocationID) \

   REFERENCES Location(LocationID)) """                         & This SQL statement will create a new table called Staff with the attributes - StaffID, FirstName, LastName, JobTitle, DepartmentID, LocationID, The primary key is StaffID and the foreign keys are DepartmentID and LocationID                       \\ \hline

"""insert into\

Staff(FirstName,LastName, JobTitle, DepartmentID) values
(‘{0}’,’{1}’,’{2}’,'{3}')
""".format(FirstName, LastName, JobTitle, DepartmentID) & This SQL statement will add 3 new records to the database. In this example it is eneterng a new staff record with attributes: FirstName, LastName, JobTitle and DepartmentID \\ \hline

"""delete from Staff\

where StaffID = ‘{3}’\

""".format(StaffID) & This statement will delete the staff member from the Staff table with the StaffID of {3} \\ \hline

\end{tabular}
\end{center}

\section{Security and Integrity of the System and Data}

\subsection{Security and Integrity of Data}

\subsection{System Security}

\section{Validation}

\section{Testing}

\begin{landscape}
\subsection{Outline Plan}

\begin{center}
    \begin{tabular}{|p{2cm}|p{5cm}|p{5cm}|p{4cm}|}
        \hline
        \textbf{Test Series} & \textbf{Purpose of Test Series} & \textbf{Testing Strategy} & \textbf{Strategy Rationale}\\ \hline
        Example & Example & Example & Example \\ \hline
    \end{tabular}
\end{center}

\subsection{Detailed Plan}

\begin{center}
    \begin{longtable}{|p{1.5cm}|p{2.5cm}|p{2.5cm}|p{2cm}|p{2cm}|p{2cm}|p{2cm}|p{2cm}|}
        \hline
        \textbf{Test Series} & \textbf{Purpose of Test} & \textbf{Test Description} & \textbf{Test Data} & \textbf{Test Data Type (Normal/ Erroneous/ Boundary)} & \textbf{Expected Result} & \textbf{Actual Result} & \textbf{Evidence}\\ \hline
        Example & Example & Example & Example & Example & Example & Example & Example \\ \hline
    \end{longtable}
\end{center}
\end{landscape}
