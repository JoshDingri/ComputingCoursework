\chapter{Testing}

\section{Test Plan}

\begin{landscape}
\subsection{Original Outline Plan}

\begin{center}
    \begin{tabular}{|p{2cm}|p{5cm}|p{5cm}|p{4cm}|}
        \hline
        \textbf{Test Series} & \textbf{Purpose of Test Series} & \textbf{Testing Strategy} & \textbf{Strategy Rationale}\\ \hline
1 & Test flow of control between the user interfaces & Top-down testing & A outline of the GUI will be developed and once working more modules will be added \\ \hline
       2 & To test validation of input data is performed correctly & Bottom-up testing & Components will be tested as they become available \\ \hline
3 & To test if the data is stored in its correct location & Black box testing & An output that the program produces will be examined for correctness and then the next output can be examined.\\ \hline
4 & To check information can be read from the database and read through the system & Bottom-up testing &  Components will be tested as they become available\\ \hline
5 & Check that the system cannot have unauthorised access & Unit testing & To test individual software components \\ \hline
6 & To check the finished system will fulfil the specifications &  Acceptance Testing & Performed by the client to check the system meets their requirements\\ \hline
7 & To test that the UI works effectively and certain shortcuts can be pressed & Black Box Testing & An output that the program produces will be examined for correctness and then the next output can be examined.\\ \hline
8 & Test algorithms to make sure that the output is correct & White Box Testing & Tests will be derived from knowledge of the program code \\ \hline

    \end{tabular}
\end{center}

The outline plan has stayed the same since the design section. There were no changes needed.


\subsection{Original Detailed Plan}

\begin{center}
    \begin{longtable}{|p{1.5cm}|p{2.5cm}|p{2.5cm}|p{2cm}|p{2cm}|p{2cm}|p{2cm}|p{2cm}|}
        \hline
        \textbf{Test Series} & \textbf{Purpose of Test} & \textbf{Test Description} & \textbf{Test Data} & \textbf{Test Data Type (Normal/ Erroneous/ Boundary)} & \textbf{Expected Result} & \textbf{Actual Result} & \textbf{Evidence}\\ \hline
        Example & Example & Example & Example & Example & Example & Example & Example \\ \hline
    \end{longtable}
\end{center}

\subsection{Changes to Detailed Plan}

\begin{center}
    \begin{longtable}{|p{1.5cm}|p{2.5cm}|p{2.5cm}|p{2cm}|p{2cm}|p{2cm}|p{2cm}|p{2cm}|}
        \hline
        \textbf{Test Series} & \textbf{Purpose of Test} & \textbf{Test Description} & \textbf{Test Data} & \textbf{Test Data Type (Normal/ Erroneous/ Boundary)} & \textbf{Expected Result} & \textbf{Actual Result} & \textbf{Evidence}\\ \hline
        Example & Example & Example & Example & Example & Example & Example & Example \\ \hline
    \end{longtable}
\end{center}

\section{Test Data}

\subsection{Original Test Data}

\subsection{Changes to Test Data}

\section{Annotated Samples}

\subsection{Actual Results}

\subsection{Evidence}

\end{landscape}

\section{Evaluation}

\subsection{Approach to Testing}

\subsection{Problems Encountered}

\subsection{Strengths of Testing}

\subsection{Weaknesses of Testing}

\subsection{Reliability of Application}

\subsection{Robustness of Application}