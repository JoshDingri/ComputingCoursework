\chapter{Analysis}

\section{Introduction}

\subsection{Client Identification}

My client is Chranj Dingri; he is 50 years old and works for Volac int. as their IT Systems Manager. Volac int. is a company that produces and markets proteins, on a daily basis the company use computers in order to update and manage databases. Chranj's job includes building servers and installing operating systems on them, he also troubleshoots colleague's requests. In doing this job it means he has a lot of experience with computers and will be able to administrate the proposed system.


Chranj helps to allocate hardware devices to all co-workers in the business, it is very hard to keep track of who has what device and all the details that are linked with that device (warranties, amount paid for the device, make and model etc.). The hardware devices that he wants to keep track of are mobile phones, tablets, laptops, PCs and printers. With the new system Chranj would like to keep all his colleagues in a database and have it show which people have which device assigned to them, he would also like to see all the details about a specific hardware device. He would like the new system to be read only for colleagues looking at their own data and have only certain people being able to add entries and update information. Having this computer based system means that Chranj can have all the data he wants stored in one place so he can simply search for a colleague and get an overview of the devices assigned to them. 

\subsection{Define the current system}

Currently the system used is a spreadsheet on Excel. A customer will fill out a form manually on paper including their name and hardware devices they require, they then give it to their manager who will sign the document and give it to the IT staff who will copy the data onto the spreadsheet. The IT staff will also order the hardware device by calling up the company's distributer, after which they will use the receipt they get emailed and enter in the remaining hardware details. The spreadsheet stores the names of staff (with details such as job title and department) as well as which hardware device they have been allocated. Line managers have to also fill out these request forms and provide their own signature, IT staff however do not need to as they will store the information they want in their heads. If someone leaves the workplace the hardware device is handed back to their manager and allocated to someone else, their record will also be removed off the spreadsheet. 

The only people currently using this spreadsheet are 3 members of the IT staff (including my client) as it has not been setup for different access rights. When wanting to search for information the built-in excel search function is used. The current system does have some fairly good ways to query and sort information (such as ascending order), drop down boxes on the form title allows the user to select which records they want to show and which they want to hide. For example ticking the "Acer" box in Make/Model will show only the records with Acer as their Make/Model. 



\subsection{Describe the problems}

A spreadsheet in Excel makes it easy to enter and read information however there are quite a few downsides with the current way of doing things. The first  big issue is that multiple people cannot access the same database to update or add data, currently each IT staff member has their own spreadsheet on the system that they have to keep transferring to each other when an update is made, this is a major issue as it causes one IT staff member to work on the spreadsheet and the others to disregard it as it is more hassle than it is worth. The spreadsheet is not maintained well, some fields are placed in the wrong columns as there are not enough to handle all the data (see figure 1.7), this makes it very confusing to read and becomes cluttered with data that does not make sense such as "First Name" having "Design Studio" instead of a person's name. The spreadsheet has got to the point where it is so unorganised Chranj says "it is seen as a chore to try to update". 

Another downside to the current system is that it is not available online to be able to login from a computer that does not have the spreadsheet already. The spreadsheet also does not have restricted access for particular people, if someone wants to view their hardware device information, they would also have full access to the spreadsheet which is not ideal as there is a security risk, my client wants read-only access for other staff and managers to be able to view their department's staff information. With an Excel spreadsheet it is not very easy to search for a particular field (for example if someone wanted to search for a specific mobile phone) especially if there is a lot of data and it is not in the correct places, if a new user wanted to find "Design Studio" they would not expect to find it in the "First Name" column.

\subsection{Section appendix}

\begin{figure}[H]
\includegraphics[width=.9\textwidth,height=.9\textheight,keepaspectratio]{Page1Interview.jpg}
\caption{Interview Questions: Page 1} \label{Page1Interview}
\end{figure}

\begin{figure}[H]
\includegraphics[width=.9\textwidth,height=.9\textheight,keepaspectratio]{Page2Interview.jpg}
\caption{Interview Questions: Page 2} \label{Page1Interview}
\end{figure}

\section{Investigation}

\subsection{The current system}

\subsubsection{Data sources and destinations}

In the current system there are three main data sources used, the colleague, the line manager and the IT staff. Colleagues will fill out a physical form with information about which hardware device they would like to be assigned to them along with their name. The form will then be given to their specific line manager who will sign the form to say whether they authorize the hardware request or not. The line manager will give this form to a member of the IT staff who will enter the data onto the Excel spreadsheet. A member of the IT staff will then phone the company's supplier and find out if the item is in stock, how much it will cost, the newest device they have available and the estimated delivery time. The company will say which hardware devices they are currently giving employees and everyone will have the same business phone (for example iPhone 5's) it is not for the IT staff to decide. The IT staff member may choose to write the estimated delivery time down somewhere incase the colleague who requested it asks how long it will take. The final step is for the IT staff to add all the new information from the supplier into the database, currently some of the information on the receipt isn't stored on the database (seen below). The spreadsheet now contains all the information about which colleague has which hardware device assigned to them.

\begin{center}
\begin{tabular}{|p{2cm}|p{3cm}|p{3.5cm}|p{2cm}|}
\hline
\textbf{Source} & \textbf{Data} & \textbf{Example Data}      & \textbf{Destination} \\ \hline
Colleague                             & First Name                         & John                                               & Form                                      \\ \hline
Colleague                             & Last Name                          & Smith                                              & Form                                      \\ \hline
Colleague                             & Department                              & Financing                                            & Form                                      \\ \hline
Colleagye 			& Job Title				& Finance Accountant				& Form			\\ \hline
Colleague                             & Location                              & Orwell                                            & Form                                      \\ \hline
Colleague                             & Hardware Device Wanted             & Phone                                              & Form                                      \\ \hline
Colleague                             & Make                               & iPhone                                             & Form                                      \\ \hline
Form                                  & Form Details                       & John, Smith, Phone, iPhone,Financing,Orwell, Finance Accountant                & Line Manager                              \\ \hline
Line Manager                          & Signature                          & Tim Richardson                                     & Form                                      \\ \hline
Form                                  & Form Details Including Signature   & John, Smith, Phone, iPhone,Financing,Orwel,  Finance Accountant, Tim Richardson & IT Staff Member                           \\ \hline
IT Staff Member 	&Hardware Device Request & iPhone & Supplier 												\\ \hline
Supplier & Devices Available & iPhone 6 Plus & IT Staff Member \\ \hline
Supplier & Cost & £400 &IT Staff Member \\ \hline
Supplier & Estimated Delivery Time & 10 Working Days & IT Staff Member \\ \hline
Supplier & Device Ordered, Cost, Warranty, Warranty Period, Estimated Delivery Time, Serial Number & iPhone 6 Plus, £400, 3 year warranty, 10 working days, 832327 & Electronic Receipt \\ \hline
Electronic Receipt  & Device Ordered, Cost, Warranty, Warranty Period, Estimated Delivery Time, Serial Number & iPhone 6 Plus, £400, 3 year warranty, 10 working days, 832327 & IT member's email \\ \hline
IT Staff Member & Estimated Delivery Time & 10 Working Days & Notes (on computer or diary) \\ \hline
IT Staff Member                      & Form Details   & John, Smith, Phone, iPhone 6 Plus,Financing,  Finance Accountant ,Orwell, & Excel Spreadsheet                                   \\ \hline
IT Staff Member                             & Model                              & 6 Plus                                             & Excel Spreadsheet                                       \\ \hline
IT Staff Member                       & Serial Number                      & 71624                                          &  Excel Spreadsheet           \\ \hline
\end{tabular}
\end{center}

\newpage
\subsubsection{Algorithms}

There are a range of real life algoritms that take place in the current system.

This algorithm is a staff member completing a hardware request form and giving it to their line manager for checking.

\begin{algorithm}[H]
\begin{algorithmic}
\State $FormCompleted  \leftarrow False$
\If{$FormCompleted \leftarrow False$}
	\State Fill out form with all required fields
	\State $FormCompleted \leftarrow True$
\Else
	\State {Give to line manager}
\EndIf
\end{algorithmic}
\end{algorithm}

This algorithm is the line manager authorising the request or not, if the request is rejected he will talk to the colleague and discuss any other devices that may be chosen instead, or if a section on the form was missed out he will give it back to be corrected.

\begin{algorithm}[H]
\begin{algorithmic}
\State$Authorize \leftarrow False$
\While{$Authorize \leftarrow False$}
	\If{Line manager has a problem with hardware request}
		\State Discuss an alternative hardware device and update form
	\Else
		\State{$Authorize \leftarrow True$}
	\EndIf
\EndWhile
\end{algorithmic}
\end{algorithm}

This is the line manager giving the request form to an IT staff member, the IT staff will check for a signature (but nothing else as this is the managers job). If there is no signature, they will simply hand it back to the manager until they sign it.

\begin{algorithm}[H]
\begin{algorithmic}
\State $Signature  \leftarrow True$
\While{$Signature \leftarrow True$}
	\State Check for line managers signature
	\If {Signature not present}
		\State Give back to line manager of their department
		\State $Signature  \leftarrow False$
	\Else
		\State Keep request form for entering details into spreadsheet
\EndIf
\EndWhile
\end{algorithmic}
\end{algorithm}

This algorithm is the process of the IT staff calling up their hardware distributers and discussing the requested device, at this point they will ask if the model the company currently uses is available, if it isn't they will discuss this with the line manager and most probably wait a few days until the device is in stock again. If it is in stock a receipt will be emailed with all details on, however it would be a good idea to note down when the order will arrive incase someone asks. The receipt details are then put on to the spreadsheet along with the details from the hardware request form.

\begin{algorithm}[H]
\begin{algorithmic}
\State $Entry_Added \leftarrow False$
\State $PhoneCall \leftarrow True$
\While{$Entry_Added \leftarrow False$}
	\While{$PhoneCall \leftarrow True$}
		\State Call up distributers
		\State Check if hardware device is in stock with the newest available model
		\If {Hardware device is not in stock}
			\State Discuss the situation with line manager
			\State  $PhoneCall \leftarrow False$
		\Else
			\State Note down delivery date
			\State  $PhoneCall \leftarrow False$
		\EndIf
	\EndWhile
	\State Purchase hardware device
	\State Input hardware request form details
	\State Input receipt details including serial and model
	\State $Entry_Added \leftarrow True$
\EndWhile
\end{algorithmic}
\end{algorithm}

This algorithm is disposing of a request form after the details have been thoroughly checked over. It is not necessary to keep lots of paper if it is already matched on the speedsheet.

\begin{algorithm}[H]
\begin{algorithmic}
\State $DisposedOfForm  \leftarrow False$
\While{$DisposedOfForm \leftarrow False$}
	\State Ensure spreadsheet matches request form
	\If {Data on form matches spreadsheet}
		\State Dispose of request form
	\Else
		\State Change data so it matches 
\State $DisposedOfForm \leftarrow True$
\EndIf
\EndWhile
\end{algorithmic}
\end{algorithm}

This algorithm is using the built in search function to find a field in the spreadsheet on Microsoft Excel. On Excel everytime you press 'Find' the program will jump to the next field matching the search, if a full cycle is made and the data is not found then there may be a spelling mistake or the data is not present.

\begin{algorithm}[H]
\begin{algorithmic}
\State$BeginSearch \leftarrow True$
\State$NotFound \leftarrow True$
\While{$BeginSearch \leftarrow True$}
	\State Use the shortcut CTRL F or locate 'Find' in 'Edit' Menu
	\State Enter field name and press find
	\While{$NotFound \leftarrow True$}
		\State Press find to cycle through data
		\If {Item is not located and cycle is at its end}
			\State Check for spelling mistakes
			\State Search again
			\State $BeginSearch \leftarrow False$
			\State $NotFound \leftarrow False$
		\Else
			\State $NotFound \leftarrow False$
		\EndIf
	\EndWhile
	\State Update or Check field
	$BeginSearch \leftarrow False$
\EndWhile

\end{algorithmic}
\end{algorithm}
		

\subsubsection{Data flow diagram}

\begin{figure}[H]
\includegraphics[width=\textwidth]{CurrentDFD.jpg}
\caption{A colleague filling in a hardware request form with all necessary details, this is then passed to the line manager for checking. The manager will verify and validate the information and either hand it back to the colleague in person or sign it and keep hold of it.} \label{Page1Interview}
\end{figure}

\begin{figure}[H]
\includegraphics[width=\textwidth]{dtd1.jpg}
\caption{This data flow diagram shows how a line manger would complete a hardware request form. Since they are the highest in their department they will need to thoroughly verify and validate the information (since no one else checks it) before signing it themselves.} \label{Page1Interview}
\end{figure}

\begin{figure}[H]
\includegraphics[width=\textwidth]{dataflowdiagram2.jpg}
\caption{The filled out form is given to an IT staff member who will check for a valid signature. If there is no signature or the signature looks fraudulant they will hand it back (in person) to the line manager who will sign it. If it is valid, the form can be stored in their physical desk draw.} \label{Page1Interview}
\end{figure}

\begin{figure}[H]
\includegraphics[width=\textwidth]{dataflowdiagram3.jpg}
\caption{This is the process of the hardware requests being purchased. The IT staff check with the company's distributer via phone to check if the item is in stock, if it is not in stock the manager is told and options are discussed to either wait for it to be back in stock or cancel the request. If the item is in stock, the information about the product is entered into the spreadsheet along with the details off the request form} \label{Page1Interview}
\end{figure}

\subsubsection{Input Forms, Output Forms, Report Formats}

\begin{figure}[H]
\includegraphics[width=.9\textwidth,height=.9\textheight,keepaspectratio]{HardwareRequestForm.jpg}
\caption{\textbf{The Hardware Request Form}:This document has only been filled in with the required details to be stored on the system. The rest of the details are not necessary to be stored in the hardware allocation system. At the bottom of the form the signature from the line manager is shown, this is therefore an authorised form.} \label{HardwareRequestForm}
\end{figure}

\begin{figure}[H]
\includegraphics[width=.9\textwidth,height=.9\textheight,keepaspectratio]{Spreadsheet2.jpg}
\caption{\textbf{The Hardware Allocation Spreadsheet}: As shown above the spreadsheet is not properly used by staff. Wrong data has been entered in the name box (see cell 172) and altogether the spreadsheet is messy. The description section on the spreadsheet displays a basic statement of what the hardware device is, for instance "HP ZR22w" (cell F 145) is classed as a "Monitor". A good feature of this spreadsheet is the dropdown arrows next to the column headings, from here you can sort data into ascending or decending order (alphabetical order) and choose to hide or show certain fields. Overall the spreadsheet does store the necessary data for each person with their hardware devices but needs to be a lot clearer.} \label{Spreadsheet}
\end{figure}

\begin{figure}[H]
\includegraphics[width=.9\textwidth,height=.9\textheight,keepaspectratio]{LeaveOfWork.jpg}
\caption{When someone leaves the company they are required to give back all hardware devices. I spoke to my client after the interview to check of there are any official forms for returning devices to the company, Chranj says there is no official form or signature required for this, so in most cases their line manager will simply write it down as shown above. After they have left they can be removed from the spreadsheet and their hardware device will be allocated to someone else getting hired (or put into storage).} \label{LeaveOfWork}
\end{figure}

\begin{figure}[H]
\includegraphics[width=.9\textwidth,height=.9\textheight,keepaspectratio]{OrdersPlaced.jpg}
\caption{When the IT staff have ordered new hardware devices from the company's distributer it may be handy to write down when the delivery will be coming. Colleagues may become impatient so it is good to have a date. They may also choose to write down how many items are being delivered incase the wrong amount arrive.} \label{OrdersPlaced}
\end{figure}

Whilst talking to Chranj after the interview I asked if there were any forms to sign when actually given the hardware device to show you accepted it. There is no signature or form when the staff member is given the actual hardware device, all staff members of the same status will get similar hardware devices and they can either accept it or use their own. If rejected, the device is allocated to somebody else and the spreadsheet can be edited so the individual is no longer allocated that device (as they are added to the spreadsheet beforehand). The whole company will use the same buisness phone (normally iPhone's or Nokia's). 

\subsection{The proposed system}

\subsubsection{Data sources and destinations}

In the proposed system collegue details and hardware requests will still be received via physical form and entered manually onto the system. The new system will store the same information from the colleagues however the IT staff will store more infomation on products such as IMEI numbers (for phones) (see question 3 in interview). The phone number will be added to the database after the phone has arrive so the colleague can insert their sim card. The IMEI number is printed on the box and will only be known when delivered this will be recorded before giving to the colleague.

\newpage

\begin{longtable}{|p{2cm}|p{3cm}|p{3cm}|p{2cm}|}
\hline
\multicolumn{1}{|c|}{\textbf{Source}} & \multicolumn{1}{c|}{\textbf{Data}} & \multicolumn{1}{c|}{\textbf{Example Data}}         & \multicolumn{1}{c|}{\textbf{Destination}} \\ \hline
Colleague                             & First Name                         & John                                               & Form                                      \\ \hline
Colleague                             & Last Name                          & Smith                                              & Form                                      \\ \hline
Colleague                             & Department                              & Financing                                            & Form                                      \\ \hline
Colleague                             & Job Title                              & Financing Manager                                          & Form                                      \\ \hline
Colleague                             & Location                              & Orwell                                            & Form                                      \\ \hline
Colleague                             & Make                              & iPhone                                             & Form                                      \\ \hline
Colleague                             & Hardware Device Wanted             & Phone                                              & Form                                      \\ \hline
Form                                  & Form Details                       & John, Smith, Phone, iPhone, Financing , Financing Manager, Orwell,            & Line Manager                              \\ \hline
Line Manager                          & Signature                          & Tim Richardson                                     & Form                                      \\ \hline
Form                                  & Form Details Including Signature   & John, Smith, Phone, iPhone, Financing , Financing Manager, Orwell, Tim Richardson & IT Staff Member                           \\ \hline
IT Staff Member 	&Hardware Device Request & iPhone & Supplier 												\\ \hline
Supplier & Devices Available & iPhone 6 Plus & IT Staff Member \\ \hline
Supplier & Cost & £400 &IT Staff Member \\ \hline
Supplier &Warranty & Yes/No & IT Staff Member \\ \hline
Supplier &Warranty Period & 3 Years & IT Staff Member \\ \hline
Supplier & Estimated Delivery Time & 10 Working Days & IT Staff Member \\ \hline
Supplier & Device Ordered, Cost, Warranty, Warranty Period, Estimated Delivery Time, Serial Number & iPhone 6 Plus, £400, 3 year warranty, 10 working days, 832327 & Electronic Receipt \\ \hline
Electronic Receipt  & Device Ordered, Cost, Warranty, Warranty Period, Estimated Delivery Time, Serial Number & iPhone 6 Plus, £400, 3 year warranty, 10 working days, 832327 & IT member's email \\ \hline
IT Staff Member & Estimated Delivery Time & 10 Working Days & Notes (on computer or diary) \\ \hline
IT Staff Manager                      & Form Details   & John, Smith, Phone, iPhone, Financing , Financing Manager, Orwell, & Database                                  \\ \hline
IT Staff Member                             & Model                              & 6 Plus                                             & Database (linked with colleague)                                      \\ \hline
IT Staff Member                       & Cost                              & £300                                               & Database (linked with colleague)          \\ \hline
IT Staff Member                       & Warranty                           & Yes                                        & Database (linked with colleague)          \\ \hline
IT Staff Member                       & Warranty Period                           & 3 Years                                            & Database (linked with colleague)          \\ \hline
IT Staff Member                       & Serial Number                      & 832327                                             & Database (linked with colleague)          \\ \hline
IT Staff Member                       & Purchase Date                      & 11/05/2015                                         & Database (linked with colleague)          \\ \hline
Colleague                       & Phone Number (if phone)            & 07xxxxxxxxx                                        & IT Staff Member          \\ \hline
Phone casing(if phone)		&IMEI Number (if phone)            & EH37000781                                      & IT Staff Member          \\ \hline	
IT Staff Member & Phone Number (if phone)            & 07xxxxxxxxx & Database (linked with colleague) \\ \hline
IT Staff Member & IMEI Number (if phone)           &EH37000781        & Database (linked with colleague) \\ \hline
\end{longtable}
\

\subsubsection{Data flow diagram}

\begin{figure}[H]
\includegraphics[width=\textwidth]{CurrentDFD.jpg}
\caption{This is the same process as with the last system. The colleague will fill in all necessary form details and give it to their line manager. The line manager wil verify and validate the information andl sign it.} \label{Page1Interview}
\end{figure}

\begin{figure}[H]
\includegraphics[width=\textwidth]{dataflowdiagram2.jpg}
\caption{This part of the system is also the same as the current system. The form is given to the IT staff who will check if the request form has a manager's signature. } \label{Page1Interview}
\end{figure}

\begin{figure}[H]
\includegraphics[width=\textwidth]{DFDLogin.jpg}
\caption{This shows hardware information being entered onto the new system. The IT Staff have their own login credentials to provide them with full access and they can then enter new data. After data has been added it will be necessary to log out again. If the item is not in stock options are discussed with the line manager (Just as figure 1.6)  } \label{Page1Interview}
\end{figure}

\begin{figure}[H]
\includegraphics[width=\textwidth]{DFDLoginColleague.jpg}
\caption{This shows a colleague logging into the system to read their own data with read-only access. } \label{Page1Interview}
\end{figure}

\begin{figure}[H]
\includegraphics[width=\textwidth]{DFDLoginManager.jpg}
\caption{This shows how managers also have read only access to check their own information just like their colleagues.} \label{Page1Interview}
\end{figure}

\begin{figure}[H]
\includegraphics[width=\textwidth]{ManagerReadOnly.jpg}
\caption{This shows how managers also have read only access to check all staff in their department.} \label{Page1Interview}
\end{figure}



\subsubsection{Data dictionary}

\begin{center}
\begin{longtable}{|p{2cm}|p{1.14cm}|p{1.1cm}|p{1.7cm}|p{1.7cm}|p{2cm}|}
\hline
\textbf{Name} & \textbf{Data Type}& \textbf{Length} & \textbf{Validation} & \textbf{Example Data} & \textbf{Comment}      \\ \hline
StaffID                             & Integer                                 & 1-200                     & Range Validation \textgreater0 and \textless=350                                 & 132                   & Unique to the staff   \\ \hline
Staff-FirstName                      & String                                  & 1-25                                 & Presence Check                           & John                  &                       \\ \hline
Staff-LastName                       & String                                  & 1-25                                 & Presence Check                           & Smith                 &                       \\ \hline
Staff-Department                     & String                                  & 1-30                                 & Length                                   & Financing             &                       \\ \hline
StaffJobTitle			& String				& 1-30			& Length			& Finance Manager		&		\\ \hline
StaffLocation                       & String                                  & 1-20                                 & Length                                   & Orwell                &                       \\ \hline
StaffHardware	& String				&1-50				& Length				& iPhone 5S         & Shows the allocation          \\ \hline 
HardwareID                          & Integer                                 & 1-200                                & Range Validation \textgreater0 and  \textless=300                      & 121                   & Unique to each device \\ \hline
Hardware Device                      & String                                  & 1-25                                 & Length                                   & Phone                 &                       \\ \hline
Hardware Make                        & String                                  & 1-25                                 & Length                                   & iPhone                &                       \\ \hline
Hardware Model                       & String                                  & 1-25                                 & Length                                   & 5S                    &                       \\ \hline
Hardware Cost                       & Float                                 & 1-2000                             & Range Validation \textgreater0 and \textless=2000                                    & £500                  &                       \\ \hline
Hardware Warranty                    & Boolean                                 &                                      & Presence Check                           & True                  &                       \\ \hline
Hardware WarrantyPeriod              & String                                  & 1-10                                 & Length                                   & 3 Years               &                       \\ \hline
Hardware SerialNumber                & String                                  & 1-50                                 & Length                                   & 12307321              &                       \\ \hline
Hardware PurchaseDate                & Date                                  &                                  & Format                                   & 11/03/2015              &                       \\ \hline
Hardware-IMEI                & String                                  &          1-50                        & Length                                   & EH37000781               & Only required for phones                      \\ \hline
Hardware PhoneNumber                & String                                  &1-11                                  & Length                                   & 07927551125              &   Only required for phones                      \\ \hline
DepartmentID &Integer				& 1-4				&Range Validation \textgreater0 and \textless=75		& 24		& Unique for each department	\\ \hline
Department Name & String				& 1-30				& Length		& Financing		&	\\ \hline
LocationID &Integer				& 1-4				&Range Validation \textgreater0 and \textless=20		& 	24	& Unique for each work location		\\ \hline
Location Name & String				& 1-30				& Length		&	Orwell	 &  \\ \hline
Location AddrLine1 & String				& 1-30				& Length		&	50 Fisher's Ln	 &  \\ \hline
Location AddrLine2 & String				& 1-30				& Length		&	Orwell	 &  \\ \hline
Location AddrLine3  & String				& 1-30				& Length		&	Royston 	 &  \\ \hline
\end{longtable}
\end{center}

\subsubsection{Volumetrics}

Initially my system will store 350 different staff records, I chose this amount because Chranj said he has around 300 staff in his company he has to store as soon as the system is available (Interview question 5). Since he said "around 300" it shows he is fairly unsure exactly how many, so 350 means that the system will provide room for anymore staff that were not counted in the interview. The company does not hire people everyday and since this database will only be storing staff members, 350 records gives a fair bit of room to enable Chranj (and other IT staff members) to get used to the system. 200 hardware devices will be able to be stored, which is definately more than enough but provides a lot of room for new devices that get released (as hardware is released fairly often) the company wish to use. Lastly the link table will be added which stores the foreign keys from hardware and staff as well as its own primary key meaning 3 integers are stored.

If each colleague on the new proposed system has 6 fields of data stored about them (see Data Dictionary) the calculations would be:


The database consists of mainly letters, ASCII is used for the conversions below:
1 Byte = 1 Letter

For staff members 5 fields will be Strings and 1 will be Integer (StaffID).


I will take the max length of all fields, for example 'StaffFirstName' has a max of 25. Which would be 25 bytes. Altogether the string fields calculate as followed:

(25 x 2) + (30 x 2) + 20 = 130 bytes. 
There are 130 bytes per record.
There are 350 records so:
130 x 350 = 45,500 bytes (45.5 KB)

The Integer field will go up to 350. Because each number over 255 is 2 bytes you can change it between each field, therefore all will be stored as 2 bytes.

350 x 2 = 700 bytes

445 bytes + 45.5KB =  45.945 KB (45.9 KB) 

45.9KB is required for the Staff Table.

The Hardware device table has the following fields:

7 fields are strings (lengths = (25 x 3) + 10 + (50 x 2) + 11 = 196 bytes)

2 fields are integer (1 has a max of 2000 (2 bytes), the other has goes up to 200 (1 byte))

1 field is Date (ISO format string is yyyy-mm-dd (14 bytes per date))

1 field is Boolean (1 byte per value)

One record has 196 + 2 + 1 + 3 + 1 = 203 bytes
200 x 203 = 40600 bytes or 40.6 KB

The location table has the following fields:

4 string fields (length = 30bytes)
1 integer (up to 20 (1 byte)

One record has 120 + 1 = 121 bytes
20 records are stored so:
121 x 20 = 2420 bytes

The department table has the following fields:

1 string field (length = 30bytes)
2 integer (up to 75 (1 byte) and 20 (1 byte)

One record has 30 + 1 + 1 = 32 bytes
75 records are stored so:
32 x 75 = 2400 bytes (2.4kb)

The link table (StaffHardware)

3 Integers (max of 200 (1 byte), 350 (2 bytes) and 350 (2 bytes)):
1 + 1 + 2 = 3 bytes per record:
350 records stored:
3 x 350 = 1050 bytes (1.05 KB)

\

\textbf{Therefore the whole database needs a minimum of (45,945 + 40,600 + 2420 + 2400 + 1050 = 92415 Bytes - 92.415) a rounded value of 92.4 KB.}


\section{Objectives}

\subsection{General Objectives}

\begin{itemize}
\item A database to show all current staff (with details such as their job title) and the hardware devices they are assigned to.
\item The database will replace the current spreadsheet and be easy to read as information will be clear and organised.
\item An easy to use database for IT staff to enter colleague data with toolbar buttons and keyboard shortcuts as they are advanced computer users and will want an quicker way of doing things.
\item The layout for colleagues to see their hardware allocations will be clear and to the point, missing out any unnecessary buttons and menus.
\item A search function will be in place to make searching for a field (such as staff member or hardware device) easy.
\end{itemize}

\subsection{Specific Objectives}

\begin{itemize}
\item The database will have one table with staff and a relationship to the table with their hardware device. This will show which hardware devices the staff are allocated and all the details of that hardware device. Staff details including department and location will be linked to the Department and Location tables.
\item The search function will allow a user to enter text and will highlight where on the page that text is.
\item Read-Only access for staff viewing their own information (with a log-in system to allow this).
\item Read-Only access for line managers wanting to see all data about staff in their department.
\item Admin rights for IT staff so they are able to view and edit all data on the database.
\item  An automatic email will be sent to IT staff members reminding them that a warranty is running out on a particular device, this email will be sent about 3 months before the end of the warranty period.
\item Users will be able to query information, for example if they wanted to show only mobile phones or if they wanted to show only specific departments.
\item A method of sending hardware request forms electronically may be introduced to elimate the need for physical copies.
\item The database may be online using the company's server, this will mean that users will be able to access the database from anywhere.
\end{itemize}

\subsection{Core Objectives}
\begin{itemize}
\item The system will provide a login system where IT staff can assign usernames to all staff and have admin rights to view all data. The database will have read-only access for colleagues viewing their own information. Managers will be able to view their departments hardware devices.
\item The system will also have an automatic email sent to IT staff members reminding them that a warranty is running out on a particular device, this email will be sent about 3 months before the end of the warranty period.
\item The system must have a search function to allow a user to find a specific field.
\item The system must have a way to query information so that a user can filter or catergorise information (such as by department or location)
\end{itemize}

\subsection{Other Objectives}

\begin{itemize}
\item It will be nice to include a method of sending hardware request forms electronically to elimate the need for physical copies. However this will only be considered if everything else is completed as it is not essential.
\item A great feature to have is the database being available online using the server the company owns. If the database was online it will be available to use from anywhere with internet connection.
\end{itemize}

\section{ER Diagrams and Descriptions}

\subsection{ER Diagram}

\begin{figure}[H]
\hspace*{-2cm}
\vspace*{5cm}
\setlength{\abovecaptionskip}{-320pt plus 3pt minus 2pt}
\includegraphics[width=1.2\textwidth]{ERDiagram.jpg}
\caption{The ER Diagram is basic. Since the IT Staff and Line Managers also need their own hardware devices they are classified under the Staff table. All staff have many to many relationships with hardware as they can have a number of phones, printers, monitors etc. at one time.} \label{ER Diagram}
\end{figure}

\subsection{Entity Descriptions}

\begin{center}
Staff  (\underline{StaffID}, StaffFirstName, StaffLastName, StaffJobTitle, \textit{DepartmentID},\\ \textit{ LocationID})
\end{center}

\

\begin{center}
Hardware  (\underline{HardwareID}, HardwareDevice, HardwareMake, HardwareModel,\\ HardwareCost, HardwareWarranty, HardwareWarrantyPeriod,\\ HardwareSerialNumber, HardwarePurchaseDate, HardwareIMEINumber, \\HardwarePhoneNumber)
\end{center}

\

\begin{center}
StaffHardware  (\underline{StaffHardwareID}, \textit{ StaffID}, \textit{ HardwareID})
\end{center}

\

\begin{center}
Department  (\underline{DepartmentID}, DepartmentName, \textit{ LocationID})
\end{center}

\

\begin{center}
Location  (\underline{LocationID}, LocationName, LocationAddrLine1, LocationAddrLine2, LocationAddrLine3)
\end{center}



\section{Object Analysis}

\subsection{Object Listing}

\begin{itemize}
\item Staff
\item Hardware
\item Department
\item Location
\end{itemize}

\subsection{Relationship diagrams}

\begin{figure}[H]
\hspace*{-1.3cm}
\vspace*{-1cm}
\includegraphics[width=1\textwidth]{RelationshipDiagram.jpg}
\caption{The relationship diagram shows how staff own many hardware devices. The staff work at one work location and that location has many different departments. Many staff are part of one department. } \label{Relationship Diagrams}
\end{figure}

\subsection{Class definitions}

\begin{figure}[H]
\hspace*{-1.3cm}
\vspace*{-1cm}
\includegraphics[width=1\textwidth]{ClassDefinitions.jpg}
\caption{The class definitions show the attributes and methods for each class. The staff class will hold unique IDs from the other classes to make the relationship.} \label{Class Definitions}
\end{figure}

\section{Other Abstractions and Graphs}

\begin{figure}[H]
\includegraphics[width=.9\textwidth,height=.9\textheight,keepaspectratio]{HardwareGraph.jpg}
\caption{The graph provides a clear representation of how all types of staff get hardware devices, no matter if they are IT Staff, Managers or their colleagues.} \label{Page1Interview}
\end{figure}

\section{Constraints}

\subsection{Hardware}

The system will be stored onto a server stored in the workplace which is perfectly capable to handle the system, the specs are as followed:

\begin{itemize}
\item HP DL360
\item Windows 2003 (can run any operating system required)
\item 4TB Hard Drive
\item 16GB RAM
\item Quad Core Processor - 2.5 GHZ
\end{itemize}

The server is stored in a server room with high ventilation and fans operated 24/7. This is a huge benefit because the system can be running at all times so people can access the database. Preferably the overall model will be client-server and each user will hav their own login for security. All users should connect using clients on local computers and will not directly access the server.

\subsection{Software}

There is no real problem here, in the interview Chranj mentioned there are "no contraints" and I can install any programs if needed (question 13). There is a lot of hard disk space so the size of installations should not be a problem, the company runs on a fibre optic network with 100mb/s download speed so any downloads should be completed quickly. Security on the network will be set by an administrator at the company, but all client's computers already have access to the server on the network.

\subsection{Time}

The only deadline is that given by the teacher on the 17th April 2015. However Chranj has said that "the earlier, the better" and will be happy to start using the system if it is finished early.

\subsection{User Knowledge}

Chranj works as an IT systems manager (see client identification) so he is perfectly qualified to deal with generic IT problems. However he has never used SQL and database software on a professional level so it will be necessary to provide him with a user manual to help with the software and to deal with any problems. The other admin staff (in IT support) have similar knowledge of computer systems and they too will need help from the manual. One IT member is an experience Visual Basic programmer and may be able to adapt to new software easier than others as he is used to using a variety of software programs, but again he will need support with SQL if he has never used it before. As for the rest of the company's staff, user knowledge may differ and it would take a lot of time to ask every staff member in the company since there is well over 100 members, most staff will use computers on a daily basis (for spreadsheet work and research) and therefore they are not inexperienced with computers but are still fairly novice and will require as much support as possible. Buttons and tooltips (to display what a button does) will be common on the database application to aid staff with their work.

\subsection{Access restrictions}

One of the objectives for this program is to provide IT staff with full access rights to add, delete and edit entries of other staff members in the company. Line managers will have rights to view all staff department (including their own) and which hardware they currently own but these will only be read-only rights. Read-only rights will be given to colleagues wanting to view their own current hardware allocations and details about them such as Serial, Model and Warranty Period, they will also be able to view their personal details such as First Name and Surname this is because staff can then check for any errors and report any faults with personal data to managers (in accordance with the Data Protection Act). The program will be made like this because it reduces any security issues and means that if a certain staff member was given a better piece of hardware (such as a iPhone) other colleagues (apart from line managers and IT staff) will not know and conflicts can be avoided.

\section{Limitations}

\subsection{Areas which will not be included in computerisation}

The physical form that colleagues will fill in to say which hardware device they would like, which the line manager then signs, will stay the same. I spoke to my client after the interview to clarify that signing a paper document is easier than signing on a computer, Chranj does not think a computerised version of this would make things any easier. The current form does not need to be redesigned as it includes all the information needed for line managers to check, also the fact that staff have used these forms for long time means they will not have to get used to another version.

\subsection{Areas considered for future computerisation}

The hardware request forms may be computerised if there is enough time, the benefit this brings is that a computerised form is harder to misplace than a physical form. There will still need to be a way of allowing managers to sign off forms, this would be done by colleagues sending the form to managers, the managers providing an "Agree" or "Disagree" statement (by buttons instead of typing), and then forwarding the form to IT staff. Another feature that may be computerised in the future is making the database available online so it can be updated in real-time and multiple people can use it at the same time. Also since it is online and the server will run at all times, it can be accessed from outside of work if needed providing there is internet connection.

\newpage
\section{Solutions}

\subsection{Alternative solutions}
\begin{center}
\begin{longtable}{|p{1.7cm}|p{2.5cm}|p{3cm}|p{3cm}|}
\hline
\textbf{Solution} & \textbf{Explanation} & \textbf{Advantages} & \textbf{Disadvanages} \\ \hline
Excel Spreadsheet & Improving the existing spreadsheet by adding new field names such as Warranty and Serial Numbers. The spreadsheet would be clearer and all data would be entered into the correct columns as there are a sufficient amount of fields. & \begin{itemize} \item Company already knows how to use it. \item Already on computer systems in workplace \item Would not require a new system as it is just an update. \end{itemize} & \begin{itemize} \item Making relationships between data  is hard \item Data is unnecessarily repeated \item This is not ideal since it would not require much work and therefore would not be a good project. \end{itemize}  \\ \hline
Microsoft Access & Using Microsoft Access to produce a database making use of primary and foreign keys in order to make relationships between tables. & \begin{itemize} \item Fairly easy to use as there is a user friendly GUI available in Access \item Access provides a  great way to link tables/data together easily \end{itemize} &  \begin{itemize} \item Programming is not used and therefore isn't an ideal project \item The company has not been trained how to use it and it may take a while to get used to  \end{itemize}  \\ \hline
Command-Line Application &		 Using a command line interface in Python to provide a text display & 		\begin{itemize} \item Command line is much faster than running GUI when producing data \item A text display is easy to program compared to a GUI display \end{itemize} & 		\begin{itemize} \item A command-line is not user friendly as there is no GUI which will make it hard for staff to learn how to use the program \item It may be hard to learn commands if there is no GUI present \item A text display is not really an upgrade from the current spreadsheet. \end{itemize} \\ \hline
Web Based Application &		Making a database that is available online from outside of work. This will be running on the server the company owns  & 		\begin{itemize} \item Since it is on a server (running 24/7) it will be available at all times from anywhere with internet connection (this also meets client's preferred needs) \item The database may be accessed by more than one person at a time making it easier to edit records or view data \item The database will be similar to the PyQt GUI interface graphically (see solution below) to make it user friendly  \end{itemize} & 		 \begin{itemize} \item The company will have to learn how to use it as they are only used to the spreadsheet \item The server can go offline if it breaks down which, until fixed, means staff cannot access the database \item This will be very hard to program as we will not learn how to do this in lesson time. \end{itemize} \\ \hline
PyQt GUI Application &		 Providing a GUI interface for the database using PyQt with graphical buttons and shortcuts & 		\begin{itemize} \item A clear GUI interface means it is easy for staff to learn how to use the new software \item We will be learning how to program with PyQt in lesson time and I have been using it since July 2014   \end{itemize} & 		 \begin{itemize} \item Unlike the online method it cannot be accessed and updated in real-time using the web, this means if multiple people want to edit the database outside of work it will be different. \item Not ideal as it does not fully reach the clients needs as Chranj wanted an online method (see solution above) \end{itemize} \\
 \hline
\end{longtable}
\end{center}

\newpage
\subsection{Justification of chosen solution}

The ideal solution is to use a web based application since the client orginally asked for a online database to be made. However due to the fact it will take a long time to learn and implement this solution is placed in the "other objectives" section. The solution I will be using is a PyQt application, I will be using PyQt in class and already have a clear understanding of how to use it which will make implementation much faster. I spoke to my client after the interview to check if PyQt without online features would suffice and he says it should be fine but if there is time it would be a bonus to allow online connections. PyQt will be better than the current spreadsheet as the multiple tables can be made with relationships between them (hardware devices, staff, department and location). With the chosen method a log-in system can be implemented and staff can be assigned different restrictions which the client requested in the interview (question 12). Shortcuts in PyQt provide a great advantage for my client, since they are advanced computer users, after they have got used to the system they will always be finding quicker ways of doing things so the use of shortcut keys are important.